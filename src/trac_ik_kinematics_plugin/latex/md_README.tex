This package provides is a Move\-It! kinematics plugin that replaces the K\-D\-L I\-K solver with the T\-R\-A\-C-\/\-I\-K solver. Currently mimic joints are {\itshape not} supported.

\subsubsection*{As of v1.\-4.\-3, this package is part of the R\-O\-S Indigo/\-Jade binaries\-: {\ttfamily sudo apt-\/get install ros-\/jade-\/trac-\/ik-\/kinematics-\/plugin}}

To use\-:


\begin{DoxyItemize}
\item Add this package and trac\-\_\-ik\-\_\-lib package to your catkin workspace.
\item Find the Move\-It! \href{http://docs.ros.org/indigo/api/pr2_moveit_tutorials/html/kinematics/src/doc/kinematics_configuration.html}{\tt kinematics.\-yaml} file created for your robot.
\item Replace {\ttfamily kinematics\-\_\-solver\-: kdl\-\_\-kinematics\-\_\-plugin/\-K\-D\-L\-Kinematics\-Plugin} (or similar) with {\ttfamily kinematics\-\_\-solver\-: trac\-\_\-ik\-\_\-kinematics\-\_\-plugin/\-T\-R\-A\-C\-\_\-\-I\-K\-Kinematics\-Plugin}
\item Set parameters as desired\-:
\begin{DoxyItemize}
\item {\itshape kinematics\-\_\-solver\-\_\-timeout} (timeout in seconds, e.\-g., 0.\-005) and {\itshape position\-\_\-only\-\_\-ik} {\bfseries A\-R\-E} supported.
\item {\itshape solve\-\_\-type} can be Speed, Distance, Manipulation1, Manipulation2 (see trac\-\_\-ik\-\_\-lib documentation for details). Default is Speed.
\item {\itshape kinematics\-\_\-solver\-\_\-attempts} parameter is unneeded\-: unlike K\-D\-L, T\-R\-A\-C-\/\-I\-K solver already restarts when it gets stuck
\item {\itshape kinematics\-\_\-solver\-\_\-search\-\_\-resolution} is not applicable here.
\item Note\-: The Cartesian error distance used to determine a valid solution is {\itshape 1e-\/5}, as that is what is hard-\/coded into Move\-It's K\-D\-L plugin.
\end{DoxyItemize}
\end{DoxyItemize}

\subsubsection*{N\-O\-T\-E\-: My understanding of how Move\-It! works from user experience and looking at the source code (though I am N\-O\-T a Move\-It! developer)\-:}

For normal operations, Move\-It! only really calls an I\-K solver for one pose (maybe multiple times if the first result is invalid due to self collisions or the like). This I\-K solution provides a joint configuration for the goal. Move\-It! already knows the {\itshape current} joint configuration from encoder info. All planning at that point is done in {\itshape J\-O\-I\-N\-T S\-P\-A\-C\-E}. Collision detection and constraint checking may use Forward Kinematics to determine the pose of any subgoal joint configuration, but the planning {\itshape I\-S N\-O\-T} being done in Cartesian space. After a joint trajectory is found, Move\-It! tries to smooth the trajectory to make it less \char`\"{}crazy looking\char`\"{}, but this does not always result in a path that is pleasing to human users.

If you don't have obstacles in your space, you may want to try the Cartesian planning A\-P\-I in Move\-It! to get \char`\"{}straight-\/line\char`\"{} motion. Move\-It's Cartesian planner {\itshape I\-S} \char`\"{}planning\char`\"{} in Cartesian space (doing lots of I\-K calls). Though really it is just performing linear interpolation in x,y,z,roll,pitch,yaw between the current and desired poses -- it isn't really \char`\"{}searching\char`\"{} for a solution. That is, Move\-It's Cartesian capability is not doing collision avoidance or replanning -- if a collision is detected, the linear interpolation \char`\"{}planning\char`\"{} immediately stops. Unfortunately, in Move\-It! this scenario still returns {\itshape True} that a trajectory was found. This is not ideal, and needs to be fixed -- I may write a new Cartesian capability plugin for Move\-It! that addresses this, but that would be outside the scope of Inverse Kinematics. 